\documentclass[12pt]{article}
\usepackage[utf8]{inputenc} % input encoding
\usepackage[T1]{fontenc} % Use an 8-bit font encoding, so that ã is a
                         % single glyph in the font. Yields better
                         % hyphenation and better cut-and-paste from pdf
\usepackage{times} % Comment this line you you want the default font, Computer Roman
% \usepackage[portuguese]{babel} % Uncomment this file you plan to write in Portuguese
\usepackage{hyperref}
\usepackage{listings}
\usepackage{graphicx}
\usepackage[section]{placeins}
\usepackage{multicol}
\usepackage{tabularx}
\usepackage{makecell}
\usepackage{amsmath}
\usepackage{tcolorbox}
\usepackage{float}
\graphicspath{ {../images/} }
\sloppy
\usepackage{geometry}

\lstset{frame=tb,
  aboveskip=3mm,
  belowskip=3mm,
  showstringspaces=false,
  columns=flexible,
  basicstyle={\small\ttfamily},
  numbers=none,
  breaklines=true,
  breakatwhitespace=true,
  tabsize=3
}

\title{Casual Language Documentation \\
  \Large Compiling Techniques 2019--2020
}
\author{
  João David\\49448
}
\date{26/06/2020}

\begin{document}
\maketitle


\section{Getting Started}
\subsection{Syntax Highlighter}
The Casual Language has a syntax highlighter extension for Visual Studio Code. This tool also has auto-complete functionality, that can be used by pressing CTRL+SPACE after writing some characters, if there are any words previously written using those characters, they will appear (e.g. variable and function names).
The extension can be installed through the "casual-syntax-highlighter-0.0.1.vsix" file.

\subsection{Running the compiler}
The casualc script accepts as argument the casual source file that will be compiled and then creates the binary file with the same name of the source file (without the .cas). 

Since this language supports imports, all the source files involved in the process must be passed as argument, so that the compiler may link them and output a single binary file, that will have the name of the first source code passed as argument.

For instance, "casualc hello.cas math.cas util.cas" will output the binary hello.
From all the source files passed to the compiler, there must be exactly one defined method with the name "main". This will be the method executed first by the binary file.

\newpage
%-------------------------------------------------------------------------

\subsection{Writing Casual code}
The Casual language has 12 built-in functions that can be used by the programmer, in order to use them, they have to be declared. The following list summarizes all existent declarable functions, they allow the user to print, create arrays, and matrixes for each data type (Int, Float, Bool and String). \\

\begin{itemize}
		\item decl printInt(i:Int):Void
        \item decl printFloat(f:Float):Void
        \item decl printBool(b:Bool):Void
        \item decl printString(s:String):Void
        \item decl new\_int\_array(size:Int):[Int]
        \item decl new\_float\_array(size:Int):[Float]
        \item decl new\_bool\_array(size:Int):[Bool]
        \item decl new\_string\_array(size:Int):[String]
        \item decl new\_int\_matrix(size:Int, size2:Int):[[Int]]
        \item decl new\_float\_matrix(size:Int, size2:Int):[[Float]]
        \item decl new\_bool\_matrix(size:Int, size2:Int):[[Bool]]
        \item decl new\_string\_matrix(size:Int, size2:Int):[[String]]       

\end{itemize}

\newpage


\section{Implementation Decisions}
\subsection{Operator Precedence}

\textbf{Precedence order.} When computing the expression $3 + 3 * 5$, the operator with higher precedence goes first, in this case its the multiplication, the expression is  evaluated as if it was $3 + (3 * 5)$. \\

\textbf{Associativity.} When an expression has two operators with the same precedence, it is evaluated according to its associativity.  For instance, the expression $72 / 2 / 3$ is computed as if it was $(72 / 2) / 3$, because the division operator has left to right associativity. Other operators are not associative, therefore, they can't share the same operand with other operators in the same level of precedence. The expression $3 < 4 >= 4$ is invalid. \\

The following table summarizes all the information regarding precedence and associativity in Casual, a higher level means higher precedence.

\begin{table}[htb]
\centering
\begin{tabular}{| c | c | c | c |} 
 \hline
Level & Operator & Description & Associativity        \\ \hline
 9    & [ ]      & access array element & left to right \\ 
      & ( )      & parentheses          &  \\ \hline\hline
          
 8    & $- $     & unary minus          & right to left \\ 
      & $!$      & unary logical NOT    & \\ \hline\hline
      
 7    & $*$      &                      & \\ 
      & /        & multiplicative       & left to right \\
      & $\%$     &                      & \\ \hline\hline
      
 6    & $+$      & additive             & left to right\\
      & $-$      &                      & \\  \hline\hline
 
 5    & $< $     &                      & \\
      & $<=$     & relational           & not associative\\
      & $> $     &                      & \\
      & $>=$     &                      &  \\  \hline\hline 
      
 4    & $==$     & equality             & left to right\\
      & $!=$     &                      & \\ \hline\hline
      
 3    & $\&\&$   & logical AND          & left to right\\   \hline\hline
 
 2    & $||$     & logical OR           & left to right\\   \hline\hline
 
 1    & $=$      & assignment           & right to left \\   \hline
 
\end{tabular}
\end{table}

\newpage
%-------------------------------------------------------------------------

\subsection{Linking Source files}
When passing more than one source file to the compiler, the type-checking must be done across all files, in order to verify that the functions invoked in a source file different from the one where they are defined, are respecting its signature. As well as, ensure that there are no function definitions with duplicate names across all source files.

The way the compiler does this is by creating an AST for each source file. When verifying each AST, the compiler starts by adding all the function definitions of the imported source files to the function context of the current AST validator, this way the compiler can type-check the function invocations within the current AST. The is done through the use of a HashMap where the key is the source file name, and the value the corresponding AST.

If no exceptions are raised, the compiler proceeds to create a new AST with all the functions defined and declared across all previous AST's (using aliasing), this new AST does not require to be validated, because this validation was done previously.

The codegen phase is done using this new AST, and the LLVM code is written to a single ".ll" file with the same name as the first source file passed as argument. Hence the importance of avoiding duplicate function names across all source files.
\bibliographystyle{plain}
\end{document}

%%% Local Variables:
%%% mode: latex
%%% TeX-master: t
%%% End:
